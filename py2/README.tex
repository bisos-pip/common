%%% -*- Mode: LaTeX; -*-
%       RCS: $Id: README.tex,v 1.1 2017-12-19 02:07:23 lsipusr Exp $

%%%#+BEGIN: bx:dblock:lcnt:warning-intro :class "memo" :langs "en+fa"

%%%#+END:

%%%#+BEGIN: bx:dblock:lcnt:header-begin :class "memo" :langs "en+fa"
%{{{ DBLOCK-header-begin

\documentclass{article}

\usepackage{hevea} 
%HEVEA\usepackage[utf8]{inputenc}

\htmlhead{
\vspace{0.4in}
}

\htmlfoot{
\bigskip
}


\usepackage{fontspec}
\setmainfont[Mapping=tex-text]{Linux Libertine O}

\usepackage{morefloats}

\usepackage{rcs}
\usepackage{makeidx}
\usepackage{supertabular}
\usepackage{lscape}
\usepackage{array} 
\usepackage{framed}
\usepackage{listings}

\usepackage{color}

\usepackage{hyperref}
\usepackage{url}

\usepackage{fancyhdr}

\usepackage{caption}

\usepackage{fontspec}
\usepackage{xltxtra}
\usepackage{xunicode}
\usepackage{bidi}

\makeatletter
\@ifpackageloaded{caption}{\input{caption-xetex-bidi.def}}{}
\makeatother

\newfontfamily{\persian}[Script=Arabic]{XB Zar}
%\newfontfamily\arabicfont[Script=Arabic,Scale=1]{B Nazanin}%
%\newfontfamily\arabicfontsf[Script=Arabic,Scale=1]{B Nazanin}%
%\newfontinstance{\persian}[Script=Arabic]{B Nazanin}

% for in-line Arabic we need R-L control
\newenvironment{fa}{\beginR\persian}{\endR}

% simple environment for R-L paragraphs
\newenvironment{faPar}
{\everypar={\setbox0\lastbox \beginR
\box0 \persian}}{}

%}}} DBLOCK-header-begin
%%%#+END:

%%%#+BEGIN: bx:dblock:lcnt:style-params :class "memo" :langs "en+fa"
\begin{comment}
*  [[elisp:(org-cycle)][| ]]  *DBLK: style-params*                                       :: [[elisp:(beginning-of-buffer)][Top]] [[elisp:(delete-other-windows)][(1)]]  [[elisp:(org-cycle)][| ]]
\end{comment}
% ===== STYLE PARAMETERS =====

\definecolor{darkred}{rgb}{0.5,0,0}
\definecolor{darkgreen}{rgb}{0,0.5,0}
\definecolor{darkblue}{rgb}{0,0,0.5}

\hypersetup{
    bookmarks=true,         % show bookmarks bar?
    unicode=false,          % non-Latin characters in Acrobat’s bookmarks
    pdftoolbar=true,        % show Acrobat’s toolbar?
    pdfmenubar=true,        % show Acrobat’s menu?
    pdffitwindow=false,     % window fit to page when opened
    pdfstartview={FitH},    % fits the width of the page to the window
    pdftitle={My title},    % title
    pdfauthor={Author},     % author
    pdfsubject={Subject},   % subject of the document
    pdfcreator={Creator},   % creator of the document
    pdfproducer={Producer}, % producer of the document
    pdfkeywords={keyword1} {key2} {key3}, % list of keywords
    pdfnewwindow=true,      % links in new window
    colorlinks=true ,       % false: boxed links; true: colored links
    linkcolor=darkblue,     % color of internal links
    citecolor=red,          % color of links to bibliography
    filecolor=darkgreen,    % color of file links
    urlcolor=darkred        % color of external links
}


\setlength{\textwidth}{6.0in}
\addtolength{\oddsidemargin}{-0.75in}
\addtolength{\evensidemargin}{-0.75in}

\topmargin      0.00 in
\textheight     8.50 in

\setlength{\textwidth}{16.5cm}
\setlength{\topmargin}{-0.3in}
\setlength{\textheight}{8.5in}
\setlength{\oddsidemargin}{0.0cm}
\setlength{\evensidemargin}{0.0cm}


\pagestyle{fancy}
\fancyhead{} % clear all header fields  
%% \fancyhead[C]{{\small  {\tt Work In Progress}}}
\renewcommand{\headrulewidth}{0pt} % no line in header area
\fancyfoot{} % clear all footer fields
%%\fancyfoot[LE,RO]{\thepage}           % page number in "outer" position of footer line
%% \fancyfoot[RE,LO]{{\tt --EARLY DRAFT DOCUMENT--\hspace{20 mm} --Reflects Work In Progress-- }}
\fancyfoot[RE,LO]{}


\parindent 0 true pc

\addtolength{\parskip}{5pt}


%%%#+END:

%%%#+BEGIN: bx:dblock:lcnt:header-end :class "memo" :langs "en+fa"
%{{{ DBLOCK-header-end

\begin{document}
%}}} DBLOCK-header-end
%%%#+END:

%%%#+BEGIN: bx:dblock:lcnt:front-begin :class "memo" :langs "en+fa"
\begin{comment}
*  [[elisp:(org-cycle)][| ]]  *DBLK: front-begin*                                       :: [[elisp:(beginning-of-buffer)][Top]] [[elisp:(delete-other-windows)][(1)]]  [[elisp:(org-cycle)][| ]]
\end{comment}

%%%#+END:

%%%#+BEGIN: bx:dblock:lcnt:copyright :class "memo" :langs "en+fa"
\begin{comment}
*  [[elisp:(org-cycle)][| ]]  *DBLK: copyright*                                       :: [[elisp:(beginning-of-buffer)][Top]] [[elisp:(delete-other-windows)][(1)]]  [[elisp:(org-cycle)][| ]]
\end{comment}

%%%#+END:

%%%#+BEGIN: bx:dblock:lcnt:front-end :class "memo" :langs "en+fa"
\begin{comment}
*  [[elisp:(org-cycle)][| ]]  *DBLK: front-end*                                       :: [[elisp:(beginning-of-buffer)][Top]] [[elisp:(delete-other-windows)][(1)]]  [[elisp:(org-cycle)][| ]]
\end{comment}

%%%#+END:

%%%#+BEGINNOT: bx:dblock:lcnt:main-begin :class "memo" :langs "en+fa"
\begin{comment}
*  [[elisp:(org-cycle)][| ]]  *DBLK: main-begin*                                       :: [[elisp:(beginning-of-buffer)][Top]] [[elisp:(delete-other-windows)][(1)]]  [[elisp:(org-cycle)][| ]]
\end{comment}

\title{bisos.common Library}

%%%#+END:

\thispagestyle{empty}


\bigskip

\section{Overview}

Common BISOS (ByStar Internet Services OS) Library contains
general purpose modules which build on UnISOS sub-layer
capabilities.


\section{Support}

For support, criticism, comments and questions; please contact the 
author/maintainer \\
\href{http://mohsen.1.banan.byname.net}{Mohsen Banan} at: \url{http://mohsen.1.banan.byname.net/contact}


\section{Documentation}

Part of ByStar Digital Ecosystem \url{http://www.by-star.net}.

This module's primary documentation is in  \url{http://www.by-star.net/PLPC/180047}


\section{Example Usage}

\begin{verbatim}
from  bisos.common import bxpBaseDir
\end{verbatim}

%%%#+BEGIN: bx:dblock:lcnt:main-end :class "memo" :langs "en+fa"
\begin{comment}
*  [[elisp:(org-cycle)][| ]]  *DBLK: main-end*                                       :: [[elisp:(beginning-of-buffer)][Top]] [[elisp:(delete-other-windows)][(1)]]  [[elisp:(org-cycle)][| ]]
\end{comment}

\end{document}

%%%#+END:

%%%#+BEGIN: bx:dblock:lcnt:latex:end-of-file :class "memo" :langs "en+fa"
%local variables:
%major-mode: latex-mode
%folded-file: nil
%fill-column: 65
%TeX-master: ""
%End:
%%%#+END:
